
\documentclass[langauge=chinese]{hustbeamer}

\usepackage{metalogo,tabu}
\usepackage{minted}
\usepackage{framed}
\usepackage{tabularx}

\definecolor{minted_bg}{rgb}{0.95,0.95,0.95}
\newminted{latex}{bgcolor=minted_bg,frame=lines,fontsize=\footnotesize}
\newminted{bash}{bgcolor=minted_bg,frame=lines,fontsize=\footnotesize}


\graphicspath{ {.}{./fig/} }

\title{\LaTeX{}使用介绍}
\author{许铖}
\institute{\email{xucheng@me.com}}
\date{2014}{4}{11}

\begin{document}

\maketitle
\PrintTOC

\section{\LaTeX{}简介}

\begin{frame}{\secname}{\textrm{\TeX/\LaTeX}历史}
\textrm{\TeX} 是一种用于生成高质量文档的软件,由斯坦福大学的高德纳教授开发。开发的动机是当时粗糙的排版水平已经影响到他的巨著《计算机程序设计艺术》的印刷质量。

\pause
原始的\textrm{\TeX}指令较为简单,通过组合可以实现更为高级的排版行为,不过这也导致对使用者的编程能力要求较高。Laslie Lamport在\textrm{\TeX}系统的基础上封装出了更加便于使用的\textrm{\LaTeX},其内含的大量语句以足够完成一定程度的文档排版。
\end{frame}

\begin{frame}{\secname}{\textrm{\TeX}几个引擎之间的关系}
在原始的\textrm{\TeX}(我们一般称之为\textrm{plain-\TeX})引擎基础上出现了很多新的引擎。如下列举了常见的几个引擎及其关系。

\pause
\begin{center}
\everyrow{\pause}
\begin{tabular}{cc}
引擎 & 扩展 \\ \hline
\textrm{\TeX} & \textrm{\LaTeX} \\
\textrm{pdf\TeX} & \textrm{pdf\LaTeX} \\
\textrm{\XeTeX} & \textrm{\XeLaTeX} \\
\textrm{\LuaTeX} & \textrm{\LuaLaTeX} 
\end{tabular}
\end{center}

\pause
除此之外,我们通常还会用到\textrm{Bib\TeX},\textrm{makeindex}等引擎。
\end{frame}

\begin{frame}{\secname}{\textrm{\LaTeX}优缺点}
\begin{columns}
\begin{column}[t]{0.5\textwidth}
\center{\HEI 优点}
\begin{itemize}[<+->]
    \item 提供专业的版面设计。
    \item 可以方便的排版数学公式。
    \item 格式与内容分离。
    \item 丰富的宏包扩展。
\end{itemize}
\end{column}
~
\begin{column}[t]{0.5\textwidth}
\center{\HEI 缺点}
\begin{itemize}[<+->]
    \item 非所见即所得,初始学习成本高。
    \item 很难用\textrm{\LaTeX}来写结构不明、组织无序的文档。
    \item 设计一个全新的版面还是十分困难的。
    \item \textrm{\LaTeX}宏包冲突问题很严重。
\end{itemize}
\end{column}
\end{columns}
\end{frame}

\begin{frame}{\secname}{\textrm{\LaTeX}环境安装}
\begin{description}
    \item[TeXLive] \url{http://tug.org/texlive/} (Windows \& Linux)
    \item[MacTeX] \url{http://tug.org/mactex/} (OS X only)
    \item[MikTeX] \url{http://miktex.org/} (Windows only)
\end{description}
\parbox[t][1cm][t]{\textwidth}{
\onslide*<2>{
    \begin{block}{}
        \textbf{不建议安装CTeX。}
    \end{block}
}
\onslide*<3>{
    \begin{block}{}
        Windows下安装TeXLive中,如果出现中途程序崩溃的情况,可以通过先安装Perl来解决。
    \end{block}
}
}
\end{frame}

\begin{frame}{\secname}{\textrm{\LaTeX}资料}
\begin{enumerate}
    \item 《一份不太简短的\textrm{\LaTeXe}介绍》\mint{bash}|texdoc lshort-zh|
    \item \url{http://en.wikibooks.org/wiki/LaTeX/}
    \item \url{http://tex.stackexchange.com/} 和 Google
\end{enumerate}
\pause
\begin{block}{texdoc命令}
    texdoc命令用于查看\textrm{\LaTeX}文档,使用方法如下:\\*
    \hspace{.8em}\texttt{texdoc\hspace{.8em}\emph{<package name>}}
\end{block}
\end{frame}

\section{\LaTeX{}使用简述}

\begin{frame}{\secname}{\textrm{\LaTeX}源文件}
\begin{description}
    \item[\texttt{.tex}] \textrm{\LaTeX}源文件
    \item[\texttt{.cls}] \textrm{\LaTeX}模板类文件
    \item[\texttt{.sty}] \textrm{\LaTeX}宏包文件
    \item[\texttt{.bib}] \textrm{Bib\TeX}参考文献索引
\end{description}
\end{frame}

\begin{frame}{\secname}{一般编译顺序}
\onslide<3->{\texttt{lualatex\hspace{.8em}\emph{<file name>}}} \\
\onslide<3->{\texttt{bibtex\hspace{.8em}\emph{<file name>}} / \texttt{makeindex\hspace{.8em}\emph{<$\dots$>}}} \\
\onslide<2->{\texttt{lualatex\hspace{.8em}\emph{<file name>}}} \\
\onslide<1->{\texttt{lualatex\hspace{.8em}\emph{<file name>}}} \\
\onslide<1->{生成文档,需运行一次\textrm{\LaTeX}。} \\
\onslide<2->{生成目录和交叉引用,需运行两次\textrm{\LaTeX}。} \\
\onslide<3->{生成参考文献或索引,需运行三次\textrm{\LaTeX}。}
\end{frame}

\begin{frame}[fragile]{\secname}{编译 --- latexmk}

\begin{bashcode}
$ latexmk -bibtex -lualatex -shell-escape <file name>
$ latexmk -c
\end{bashcode}

\end{frame}

\begin{frame}[fragile]{\secname}{语法基础 --- 空格和空行}
\textrm{\LaTeX}将超过一个的连续空格只会被认为是一个空格,同时对单个回车忽略不计。
换行使用一个或多个连续空行表示换行,或者你可以使用\textbackslash{}\textbackslash{}来强制换行。

\begin{columns}
\begin{column}[b]{0.5\textwidth}
\begin{latexcode}
It does not matter whether
you enter one or several 
spaces        after a word.

An empty line starts a new
paragraph.
\end{latexcode}
\end{column}
~
\begin{column}[b]{0.5\textwidth}
\footnotesize\setlength{\parindent}{2em}
\begin{framed}
It does not matter whether
you enter one or several 
spaces        after a word.

An empty line starts a new
paragraph.
\end{framed}
\end{column}
\end{columns}
\end{frame}

\begin{frame}[fragile]{\secname}{语法基础 --- 特殊字符}
\onslide<2->{特殊字符字符通过插入反斜杠来获得。}\onslide<3->{但有一个例外:\textbackslash}
\begin{columns}
\begin{column}[b]{0.5\textwidth}
\onslide<2->{
\footnotesize\ttfamily
\begin{framed}
\textbackslash\# \textbackslash\$ \textbackslash\% \textbackslash\^{} \textbackslash\& \textbackslash\_ \textbackslash\{ \textbackslash\} \textbackslash\~{} \onslide<3->{\textbackslash{}textbackslash}
\end{framed}
}
\end{column}
~
\onslide<1->{
\begin{column}[b]{0.5\textwidth}
\footnotesize\setlength{\parindent}{2em}
\begin{framed}
\# \$ \% \^{} \& \_ \{ \} \~{} \textbackslash
\end{framed}
\end{column}
}
\end{columns}
\end{frame}

\begin{frame}[fragile]{\secname}{语法基础 --- 命令与环境}
命令示例
\begin{latexcode}
\command \command{} \command{arg} 
\command{arg1}{arg2}
\command[opt arg]{arg}
\end{latexcode}

\pause 环境示例
\begin{latexcode}
\begin{env}
<content>
\end{env}
\end{latexcode}
\end{frame}

\begin{frame}[fragile]{\secname}{文件结构 --- 简单示例}
\begin{latexcode}
\documentclass{article}
\begin{document}
Small is beautiful.
\end{document}
\end{latexcode}
\end{frame}

\begin{frame}[fragile]{\secname}{文件结构 --- 稍复杂例子}
\begin{latexcode}
\documentclass[12pt,a4paper]{article}
\usepackage{hyperref}
\title{An Example}
\author{Xu Cheng}
\date{}
\begin{document}
\maketitle
Equation~\ref{eq:1} is below:
\begin{equation}
 E = mc^2 \label{eq:1}
\end{equation}
\end{document}
\end{latexcode}
\end{frame}

\begin{frame}[fragile]{\secname}{分层命令}
如下命令用于分层。
\begin{latexcode}
\part{}
\chapter{}
\section{}
\subsection{}
\subsubsection{}
\tableofcontents
\end{latexcode}
\end{frame}

\begin{frame}[fragile]{\secname}{罗列环境}
一共有三种罗列环境。
\begin{columns}
\begin{column}[b]{0.5\textwidth}
\begin{latexcode}
\begin{enumerate}
    \item Item 1
    \item Item 2
\end{enumerate}
\begin{itemize}
    \item Item 1
    \item Item 2
\end{itemize}
\begin{description}
    \item[key] value
\end{description}
\end{latexcode}
\end{column}
~\pause
\begin{column}[b]{0.5\textwidth}
\footnotesize\setlength{\parindent}{2em}
\begin{framed}
\begin{enumerate}
    \item Item 1
    \item Item 2
\end{enumerate}
\begin{itemize}
    \item Item 1
    \item Item 2
\end{itemize}
\begin{description}
    \item[key] value
\end{description}
\end{framed}
\end{column}
\end{columns}
\end{frame}

\begin{frame}[fragile]{\secname}{}
罗列环境可以相互嵌套。
\begin{columns}
\begin{column}[b]{0.5\textwidth}
\begin{latexcode}
\begin{enumerate}
    \item Level 1
    \begin{enumerate}
        \item Level 2
    \end{enumerate}
\end{enumerate}
\end{latexcode}
\end{column}
~
\begin{column}[b]{0.5\textwidth}
\footnotesize\setlength{\parindent}{2em}
\begin{framed}
\begin{enumerate}
    \item Level 1
    \begin{enumerate}
        \item Level 2
    \end{enumerate}
\end{enumerate}
\end{framed}
\end{column}
\end{columns}
\end{frame}

\begin{frame}[fragile]{\secname}{字体切换}
\footnotesize
\begin{tabular}{@{}rl@{\qquad}rl@{}}
\verb|\textrm{...}|        &      \textrm{roman}&
\verb|\textsf{...}|        &      \textsf{sans serif}\\
\verb|\texttt{...}|        &      \texttt{typewriter}\\[6pt]
\verb|\textmd{...}|        &      \textmd{medium}&
\verb|\textbf{...}|        &      \textbf{bold face}\\[6pt]
\verb|\textup{...}|        &      \textup{upright}&
\verb|\textit{...}|        &      \textit{italic}\\
\verb|\textsl{...}|        &      \textsl{slanted}&
\verb|\textsc{...}|        &      \textsc{small caps}\\[6pt]
\verb|\emph{...}|          &            \emph{emphasized} &
\verb|\textnormal{...}|    &      \textnormal{document font} 
\end{tabular}
\end{frame}

\begin{frame}[fragile]{\secname}{字体切换}
\begin{tabular}{@{}ll}
\footnotesize \verb|\tiny|         & \tiny        tiny font \\
\footnotesize \verb|\scriptsize|   & \scriptsize  very small font\\
\footnotesize \verb|\footnotesize| & \footnotesize  quite small font \\
\footnotesize \verb|\small|        &  \small            small font \\
\footnotesize \verb|\normalsize|   &  \normalsize  normal font \\
\footnotesize \verb|\large|        &  \large       large font
\end{tabular}%

\begin{tabular}{ll@{}}
\footnotesize \verb|\Large|        &  \Large       larger font \\[5pt]
\footnotesize \verb|\LARGE|        &  \LARGE       very large font \\[5pt]
\footnotesize \verb|\huge|         &  \huge        huge \\[5pt]
\footnotesize \verb|\Huge|         &  \Huge        largest
\end{tabular}
\end{frame}

\begin{frame}[fragile]{\secname}{数学公式}
\begin{columns}
\begin{column}[b]{0.5\textwidth}
\begin{latexcode}
$\lim_{n \to \infty}
\sum_{k=1}^n \frac{1}{k^2} 
= \frac{\pi^2}{6}$

$$
\lim_{n \to \infty}
\sum_{k=1}^n \frac{1}{k^2}
= \frac{\pi^2}{6}
$$
\end{latexcode}
\end{column}
~
\begin{column}[b]{0.5\textwidth}
\footnotesize\setlength{\parindent}{2em}
\begin{framed}
$\lim_{n \to \infty}
\sum_{k=1}^n \frac{1}{k^2} 
= \frac{\pi^2}{6}$

$$
\lim_{n \to \infty}
\sum_{k=1}^n \frac{1}{k^2}
= \frac{\pi^2}{6}
$$
\end{framed}
\end{column}
\end{columns}
\end{frame}

\begin{frame}[fragile]{\secname}{表格}
\begin{footnotesize}
\begin{latexcode}
\usepackage{multirow}

\begin{tabular}{|c|c|c|c|c|}
\hline
label 2-1 & label 2-2 & label 3-3 & label 4-4 & label 5-5 \\
\hline
\multirow{2}{*}{Multi-Row} & \multicolumn{2}{|c|}{Multi-Column}
& \multicolumn{2}{|c|}{\multirow{2}{*}{Multi-Row and Col}} \\
\cline{2-3}
& column-1 & column-2 & \multicolumn{2}{|c|}{}\\
\hline
\end{tabular}

\end{latexcode}
\end{footnotesize}
\end{frame}

\begin{frame}{\secname}{表格}
\begin{table}
\centering
\begin{tabular}{|c|c|c|c|c|}
\hline
label 2-1 & label 2-2 & label 3-3 & label 4-4 & label 5-5 \\
\hline
\multirow{2}{*}{Multi-Row} & \multicolumn{2}{|c|}{Multi-Column} & \multicolumn{2}{|c|}{\multirow{2}{*}{Multi-Row and Col}} \\
\cline{2-3}
& column-1 & column-2 & \multicolumn{2}{|c|}{}\\
\hline
\end{tabular}
\end{table}
\end{frame}

\begin{frame}[fragile]{参考文献}
使用\textrm{Bib\TeX}宏包可以方便的处理参考文献。\\
\pause
一般来说会将需要引用的文献按照特定语法保存为一个bib文件。
\begin{latexcode}
@BOOK{TEXGURU99,
  AUTHOR        = "{\TeX}Guru",
  TITLE         = "{\LaTeXe} 用户手册",
  YEAR          = "1999"
}
\end{latexcode}
\pause
在需要时使用$\backslash$cite指令,如
\mint[bgcolor=minted_bg]{latex} |\cite{TEXGURU99}|
\pause
在文章结束时使用$\backslash$bibliography来生成参考文献,如
\mint[bgcolor=minted_bg]{latex} |\bibliography{refs}|
\end{frame}

\section{模板介绍}

\begin{frame}[fragile]{\secname}{模板地址}
这些是我制作的模板:

\noindent\url{https://github.com/xu-cheng/hust-latex-template}

\begin{description}
\footnotesize
    \item[hustthesis] \url{https://github.com/hust-latex/hustthesis}
    \item[hustreport] \url{https://github.com/hust-latex/hustreport}
    \item[hustbeamer] \url{https://github.com/hust-latex/hustbeamer}
    \item[husttrans] \url{https://github.com/hust-latex/husttrans}
    \item[itecreport] \url{https://github.com/hust-latex/itecreport} 
\end{description}

\begin{block}{提醒}
使用前请阅读模板文档及示例。
\end{block}
\end{frame}

\begin{frame}[fragile]{\secname}{模板安装}

{\Large 使用模板环境要求:}
\begin{description}
    \item[\LaTeX{}环境] 安装最新版本的 TeXLive(推荐)或 MiKTeX,\newline\textbf{不要}使用CTeX。确保所有宏包都更新至最新。
    \item[字体] 安装如下中文字体:\\
    \begin{itemize}
        \item AdobeSongStd-Light
        \item AdobeKaitiStd-Regular
        \item AdobeHeitiStd-Regular
        \item AdobeFangsongStd-Regular
    \end{itemize}
\end{description}

\end{frame}

\begin{frame}[fragile]{\secname}{模板安装}
安装模板使用如下命令:
\mint[bgcolor=minted_bg]{bash} |make install|
使用如下命令卸载:
\mint[bgcolor=minted_bg]{bash} |make uninstall|

对于没有安装Make的Windows用户,安装命令对于如下:
\mint[bgcolor=minted_bg]{bat} |makewin32.bat install|
卸载命令如下:
\mint[bgcolor=minted_bg]{bat} |makewin32.bat uninstall|
\end{frame}

\begin{frame}[fragile]{\secname}{模板使用}

在源文件开头处选择加载文档类型,即可使用本模板,如:

\begin{latexcode}
\documentclass[options..]{hustthesis}
\documentclass[options..]{hustreport}
\documentclass[options..]{hustbeamer}
\end{latexcode}

\end{frame}

\begin{frame}[fragile]{\secname}{模板使用}

加载模板时,可用选项如下:\\[1.2ex]

\begin{tabularx}{.8\textwidth}{|l|X|}
\hline
\multicolumn{1}{|c|}{\textbf{模板}} & \multicolumn{1}{c|}{\textbf{可用选项}} \\ \hline
hustthesis & language,format,degree \\ \hline
hustreport & language,format,category \\ \hline
hustbeamer & language \\ \hline
\end{tabularx}

\end{frame}

\begin{frame}[fragile]{\secname}{模板使用示例}
\begin{latexcode*}{fontsize=\tiny}
\documentclass[degree=phd,language=chinese]{hustthesis}

\stuno{你的学号}
\schoolcode{10487}
\title{中文标题}{英文标题}
\author{作者名}{作者名拼音}
\major{专业中文}{专业英文}
\supervisor{指导老师中文}{指导老师英文}
\date{2013}{7}{1} % 答辩日期

\zhabstract{中文摘要}
\zhkeywords{中文关键字}
\enabstract{英文摘要}
\enkeywords{英文关键字}
\end{latexcode*}
\end{frame}

\begin{frame}[fragile]{\secname}{模板使用示例}
\begin{latexcode*}{fontsize=\tiny}
\begin{document}
\frontmatter
\maketitle
\makeabstract
\tableofcontents
\listoffigures
\listoftables

\mainmatter
%% 正文

\backmatter
\begin{ack}
%% 致谢
\end{ack}
\bibliography{参考文献.bib文件}
\appendix
\begin{publications}
%% 发表过的论文列表
\end{publications}
%% 附录剩余部分
\end{document}
\end{latexcode*}
\end{frame}

\section{常用宏包介绍}

\begin{frame}[fragile]{\secname}{中文字体 --- luatexja}
\begin{description}
    \item[宏包] luatexja
    \item[文档] \mint{bash}|texdoc luatexja-zh| 
    \item[注释] \LuaTeX{}下的宏包。如果你使用\XeLaTeX{},有类似宏包xeCJK。\pause
    \item[示例] \hfill\\
\begin{latexcode*}{fontsize=\tiny}
\usepackage{luatexja-fontspec}
% 英文字体
\setmainfont[Ligatures={Common,TeX}]{Tex Gyre Termes}
\setsansfont[Ligatures={Common,TeX}]{Droid Sans}
\setmonofont{CMU Typewriter Text}
\defaultfontfeatures{Mapping=tex-text,Scale=MatchLowercase}
% 中文字体
\setmainjfont[BoldFont={AdobeHeitiStd-Regular},
              ItalicFont={AdobeKaitiStd-Regular}]
             {AdobeSongStd-Light}
\setsansjfont{AdobeKaitiStd-Regular}
\defaultjfontfeatures{JFM=kaiming}
\end{latexcode*}
\end{description}
\end{frame}

\begin{frame}[fragile]{\secname}{算法环境 --- algorithm2e}
\begin{description}
    \item[宏包] algorithm2e
    \item[文档] \mint{bash}|texdoc algorithm2e| \pause
    \item[示例] \hfill\\
\begin{latexcode*}{fontsize=\tiny}
\begin{algorithm}[H]
\SetAlgoLined
\KwData{this text}
\KwResult{how to write algorithm with \LaTeX2e }
initialization\;
\While{not at end of this document}{
read current\;
\eIf{understand}{
go to next section\;
current section becomes this one\;
}{
go back to the beginning of current section\;
}
}
\caption{How to write algorithms}
\end{algorithm}
\end{latexcode*}
\end{description}
\end{frame}

\begin{frame}[fragile]{\secname}{算法环境 --- algorithm2e}
\scalebox{0.65}{
\begin{algorithm}[H]
\SetAlgoLined
\KwData{this text}
\KwResult{how to write algorithm with \LaTeX2e }
initialization\;
\While{not at end of this document}{
read current\;
\eIf{understand}{
go to next section\;
current section becomes this one\;
}{
go back to the beginning of current section\;
}
}
\caption{How to write algorithms}
\end{algorithm}    
}
\end{frame}

\begin{frame}[fragile]{\secname}{定理证明环境 --- ntheorem}
\begin{description}
    \item[宏包] ntheorem
    \item[文档] \mint{bash}|texdoc ntheorem|
    \item[注释] 使用时,需仔细阅读文档,避免宏包冲突。 \pause
    \item[示例] \hfill\\
\begin{latexcode*}{fontsize=\tiny}
\usepackage[amsmath,amsthm,thmmarks,hyperref,thref]{ntheorem}
\theoremstyle{definition}
\newtheorem{definition}{定义}[chapter]
\theoremstyle{plain}
\newtheorem{theorem}{定理}[chapter]

\begin{definition}
This is a definition.
\end{definition}
\begin{theorem}
This is a theorem.
\end{theorem}
\end{latexcode*}
\end{description}
\end{frame}

\begin{frame}[fragile]{\secname}{定理证明环境 --- ntheorem}
\begin{definition}
This is a definition.
\end{definition}
\begin{theorem}
This is a theorem.
\end{theorem}
\end{frame}

\begin{frame}[fragile]{\secname}{代码高亮 --- minted}
\begin{description}
    \item[宏包] minted
    \item[文档] \mint{bash}|texdoc minted| 
    \item[注释] 需安装Python和Pygments;编译文档时需要加上参数\verb+-shell-escape+。如:
    \mint{bash}|lualatex -shell-escape input_file|\pause
    \item[示例] \hfill\\
\begin{latexcode*}{fontsize=\tiny}
\begin{minted}[mathescape,linenos,frame=lines]{csharp}
  string title = "This is a Unicode π in the sky"
  /*
  Defined as $\pi=\lim_{n\to\infty}\frac{P_n}{d}$ where $P$ is the perimeter
  of an $n$-sided regular polygon circumscribing a
  circle of diameter $d$.
  */
  const double pi = 3.1415926535
\end{minted}
\end{latexcode*}
\end{description}
\end{frame}

\begin{frame}[fragile]{\secname}{代码高亮 --- minted}
\begin{minted}[mathescape,linenos,frame=lines]{csharp}
  string title = "This is a Unicode π in the sky"
  /*
  Defined as $\pi=\lim_{n\to\infty}\frac{P_n}{d}$ where $P$ is the perimeter
  of an $n$-sided regular polygon circumscribing a
  circle of diameter $d$.
  */
  const double pi = 3.1415926535
\end{minted}
\end{frame}

\begin{frame}[fragile]{\secname}{图片相关宏包}
\begin{center}
\begin{tabularx}{\textwidth}{|l|X|X|}
\hline
\multicolumn{1}{|c|}{\textbf{宏包}} & \multicolumn{1}{c|}{\textbf{作用}} & \multicolumn{1}{c|}{\textbf{注释}} \\ \hline
subcaption & 插入多幅并列的图片 & 不要使用subfigure,subfig这样被废弃的宏包 \\ \hline
overpic  & 在图上层叠其他内容 & \\ \hline
xypic     & 绘制简单流程图 & \\ \hline
tikz/pgf  & 更高级的绘制图片宏包 & 学习难度较大,但绘图质量高 \\ \hline
\end{tabularx}
\end{center}

\pause
\begin{block}{提醒}
插入图片时,请尽量使用矢量图(pdf,eps等格式)。Matlab,matplotlib(Python),R,visio等绘图工具都有相应导出设置。
\end{block}
\end{frame}

\begin{frame}[fragile]{\secname}{表格相关宏包}
\begin{center}
\begin{tabularx}{\textwidth}{|l|X|X|}
\hline
\multicolumn{1}{|c|}{\textbf{宏包}} & \multicolumn{1}{c|}{\textbf{作用}} & \multicolumn{1}{c|}{\textbf{注释}} \\ \hline
tabularx & 更方便调整表格列距 &  \multirow{2}{*}{\tabincell{l}{宏包ltxtable\\合并了这两个功能}} \\ \cline{1-2}
longtable & 插入超长(跨页)表格 & \\ \hline
slashbox  & 在表格中插入斜线 & \\ \hline
excel2latex  & 将excel表格转成latex代码 & 这不是宏包,是一个工具\\ \hline
\end{tabularx}
\end{center}
\end{frame}

\begin{frame}[fragile]{\secname}{其他常用宏包}
\begin{center}
\begin{tabularx}{\textwidth}{|l|X|X|}
\hline
\multicolumn{1}{|c|}{\textbf{宏包}} & \multicolumn{1}{c|}{\textbf{作用}} \\ \hline
enumitem & 自定义列表环境的式样 \\ \hline
fancyhdr & 定义页眉页脚 \\ \hline
fancynum & 将大数每三位断开 \\ \hline
natbib & 定义参考文献格式 \\ \hline
zhnumber & 生成中文数字 \\ \hline
\end{tabularx}
\end{center}
\end{frame}
 

\begin{frame}
\centering
\Huge{谢\hspace{1em}谢!}
\end{frame}

\end{document}
